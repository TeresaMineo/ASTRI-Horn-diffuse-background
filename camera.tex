\section{ASTRI-Horn telescope} 
The ASTRI-Horn telescope is based on a dual mirror system in Schawrzschild-Couder configuration
with a focal length of 2.15 m and a wide field of view (FoV) of 9.6$^\circ$. 
The primary mirror, 4.3 m diameter, is segmented in 18 hexagonal panels, 
while the secondary mirror, 2.2 m diameter, is a monolitic hemispherical thick glass shell. 
Dedicated measurements  with an optical CCD camera showed that the
point spread function (PSF) has a constant width of a few arcmin
over the whole FoV and that  90\% of the PSF  is containerd in one camera pixel (ref). 

The focal surface camera, has a convex-shaped structure where
photon detection modules (PDMs), that are square flat modules, are symmetrically placed
with angles respect to the telescope axis 
opportunely chosen in order to follow the curvature of the focal plane.
Each PDM, composed of 8$\times$8 side-by-side Silicon photomultipliers (SiPMs), 
is installed on a printed circuit board that hosts the front-end electronics (FEE) composed by
 two application specific integrated circuits (ASIC) for the SiPM signal read out.
 The ASTRI camera FEE electronics \cite{Sottile2016} represents an innovative solution, being based on a custom peak-detector operation mode to acquire the SiPM pulses rather than the sampling technique usually adopted by other Cherenkov telescopes.
This FEE-FPGA manages the the generation of local trigger, a topological one, activated when a given number of contiguous pixels within a PDM presents a signal above a given photo-electron threshold \cite{Sottile2016}.
The back-end electronics (BEE) is the main elaboration unit of the camera which
controls and manages the overall system, including data, and
all ancillaries used to perform operations as
the camera thermal regulation, the voltage distribution management and the time events stamping.
The BEE also provides the functions necessary to process and transmit the event data as 
obtained by the FEE to an external data acquisition 
workstation responsible for receiving and storing the data packets \cite{Sottile2016}.
In order to protect the camera sensors from the external
atmospheric environment, an optical-UV transparent poly methyl methacrylate (PMMA)  window 
is mounted onto the focal surface support structure covering all the PDMs.
This window is modeled with the same radius of curvature of the focal surface \cite{Catalano2018}.
The ASTRI SST-2M camera is also equipped  with a light-tight lid to prevents accidental sunlight 
exposure of the focal surface detectors.
The camera is thermally controlled to keep the working temperature on the focal plane within the range 13-17$^\circ$C
for maintaing a goopd gain stability.


ASTRI-Horn read out electronics is AC-coupled to the detector output, and any slow varying signal is  blocked.
The camera is then blind to the diffuse night sky background (NSB) or to the light from stars in the FoV.   However, considering that sky photons actually arrive with a random Poissonian time distribution, the fluctuations generated in the electronic signal are detected as noise. ASTRI-Horn  background is then caracterised by signals distributed around an constant average value (pedestal) with fluctuations whose standard deviation is due both to the intrinsic electronic noise and to the NSB flux.  The total standard deviation of the background in the camera is obtained by the following formula:

\begin{equation}
\sigma^2=\sigma^2_{dk}+\sigma^2_{sky}
\end{equation}

\noindent
where $\sigma_{dk}$ is the intrinsic standar deviation of the electronics plus the detector noise observed in complete dark condition, as with the lid closed, and $\sigma_{sky}$ is the standard deviation induced by the  NSB  which is directly proportional to the photon flux.

