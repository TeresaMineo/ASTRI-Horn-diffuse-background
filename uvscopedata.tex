\newpage
\section{UVscope data reduction and analysis} (Cettina)
\label{sect:uvscopedata}

The data acquired by UVscope are registered, pixel by pixel, as number of counts in a pre-selected integration time; in our case we have set a rate of one acquisition each second. The conversion from counts to physical units is related to several features (geometrical factor, dark counts, average efficiency, gain uniformity) that characterize the sensor and its configuration, as widely detailed in \citep{Maccarone2020} and briefly reported in the following.  The geometrical factor, depending on the effective pixel area, pupil area and distance between pupil and photocathode, whose values have been indicated in a previous paragraph, corresponds to about 1.225~mm$^2{\times}$deg$^2$. The dark counts depend from the temperature but in our case, being UVscope thermally stabilized by the Peltier system, such noise has negligible values( 0.7~counts/second per pixel). The mean value of the global average efficiency of the sensor (comprehensive of collecting, trigger and quantum efficiencies) was obtained by NIST-calibration in lab (11.85\%) and adjusted for the quartz window, 99\% UV transparent , present in UVscope mounted aboard ASTRI-Horn. The gain uniformity map, pixel per pixel, was derived from periodic “flat field” acquisitions obtained by putting a fluorescent paper in the inner part of the collimator cup so to uniformly illuminate the UVscope sensor. Nevertheless, despite the gain equalization, a difference in efficiency of the order of 50\% is always present between the internal 36 pixels and those along the external perimeter of the sensor; this is due to the extension of the photocathode, such that the photoelectrons emitted on the edge are focused on the outermost pixels. For such a reason, the 28 pixels of the external perimeter are not considered in the analysis. Last but not least, in the evaluation of the diffuse NSB flux, it is necessary to identify the maximum number of pixels on which a bright star could appear in UVscope; a pre-analysis of the data confirmed that such a star appears in a maximum of 4 pixels, if the star is located across pixels. The final number of pixels (32) used for the evaluation of the diffuse NSB forms what is hereafter called ‘useful mask’ dynamically identified acquisition by acquisition having eliminated the 28 pixels along the external perimeter and the first 4 pixels with highest content as detected in the inner 6$\times$6 pixels region.

In brief,  to evaluate the mean flux of the diffuse NSB, \textit{$<NSB>(t)$}, the UVscope data, pixel by pixel, are cleaned of dark counts (albeit very low), normalized to the gain uniformity map of its sensor unit, scaled for the global average efficiency of the sensor itself and, after selection in the useful mask, used for the evaluation of:\\

\noindent
\[<NSB>\left(t\right)=\frac{{CTS}^*(t)\cdot {10}^{-3}}{{GF}_{pixel}\cdot <{\varepsilon }_{total}>}\ \ \ \ \ \ \ \ \ \ \left[\frac{photons}{m^2\cdot ns\cdot sr}\right]\]
or
\[<NSB>\left(t\right)=\frac{{CTS}^*(t)\cdot {10}^{-3}}{{3282.778}\cdot {GF}_{pixel}\cdot <{\varepsilon }_{total}>}\ \ \ \ \ \ \ \ \ \ \left[\frac{photons}{m^2\cdot ns\cdot deg^2}\right]\]
where:
\[{CTS}^*\left(t\right)=\ \frac{1}{Npix}\ \sum^{Npix}_{k=1}{\frac{CTS\left(k,t\right)-<dark(T\left(t\right))>}{equ(k)}}\ \ \ \ \left[\frac{counts}{second}\ per\ pixel\right]\]

\noindent where \textit{k} indicates each of the \textit{N${}_{pix}$ }forming the useful mask, \textit{CTS${}_{k}$} are the counts in the pixel \textit{k} at time \textit{t}, \textit{$<dark(T(t))>$} is the average value of the dark counts in each pixel at temperature \textit{T} at time \textit{t}, \textit{equ (k)} is the sensor gain equalization map for the period in question and whose average value of global efficiency is described by \textit{$<\varepsilon{}_{total}>$}, and  \textit{GF${}_{pixel}$} is the geometric factor of the sensor pixels.
